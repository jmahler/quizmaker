
\documentclass[fleqn]{article}
\usepackage{vmargin}
\usepackage{amsmath}
\parindent=0in

\begin{document}

\begin{flushleft}
math31-series-01 (1)\\
\today
\end{flushleft}


1).  Does it converge?  And if so find the range of values.\\

\begin{align*}
\sum_{n=1}^{\infty} \left(3*x\right)^n\\
\end{align*}


\begin{verbatim}





\end{verbatim}

2).  Does it converge?  And if so find the exact value.\\

\begin{align*}
\sum_{n=1}^{\infty} \frac{4}{2^n}
\end{align*}


\begin{verbatim}





\end{verbatim}

3).  Does it converge?  And if so find the range of values.\\

\begin{align*}
\sum_{n=1}^{\infty} \frac{x^n}{2^n}
\end{align*}


\begin{verbatim}





\end{verbatim}

4).  Does the series converge?\\

\begin{align*}
\frac{\ln2}{2} + \frac{\ln3}{3} + \frac{\ln4}{4} + \frac{\ln5}{5} + + +\\
\end{align*}

\begin{verbatim}





\end{verbatim}

5).  Does it converge?  And if so find the exact value.\\

\begin{align*}
\sum_{n=1}^{\infty}\frac{1}{n(n+3)}
\end{align*}


\begin{verbatim}





\end{verbatim}

6).  Does it converge?  And if so find the exact value.\\

\begin{align*}
\sum_{n=1}^{\infty} \ln(\frac{n+1}{n})
\end{align*}


\begin{verbatim}





\end{verbatim}

7).  Does it converge?  And if so find the exact value.\\

\begin{align*}
\sum_{n=1}^{\infty} \left(1+\frac{k}{n}\right)^{n}\\
when \;\; k 
eq 0
\end{align*}


\begin{verbatim}





\end{verbatim}

8).  Does the series converge?\\

\begin{align*}
\frac{1}{2} + \frac{1}{5} + \frac{1}{10} + \frac{1}{17} + \frac{1}{26} + + +\\
\end{align*}

\begin{verbatim}





\end{verbatim}

\end{document}
