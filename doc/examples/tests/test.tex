% This file was generated by qzm-view_test

%\documentclass{exam}
\documentclass[noanswers]{exam}

\usepackage{multicol}
\usepackage{ulem}
\usepackage{graphicx}
\usepackage{amsmath}

\parindent=0pt
%\usepackage{graphicx}

\begin{document}


% choose one or the other,
% or leave configured by \documentclass
%\printanswers
%\noprintanswers


\begin{tabular}{l r}
Name: \uline{\hspace{200pt}} \hspace{160pt} & MATH 40 \\
      & Spring 2011 \\
      & practice \\
\end{tabular}


\vspace{10pt}

Solve each problem in the space given.
No calculators are allowed.

\vspace{5pt}

\begin{questions}

\question Solve the ODE:
\begin{align*}
    (x \cdot y - 1) \; dx + (x^2 - x \cdot y) \; dy = 0
\end{align*}

\vspace*{17em}

\begin{solution}
It is not separable and it is not exact (see below) but it is close.
\begin{align*}
    \frac{\partial{M}}{\partial{y}} &= x \ne \frac{\partial{N}}{\partial{x}}
        = 2x - y
\end{align*}

Try to find a function in terms of $x$ only.
\begin{align*}
    \frac{ \frac{\partial{M}}{\partial{y}} - \frac{\partial{N}}{\partial{x}} }
        {N}
        &= \frac{x - 2x + y}{x^2 - xy} \\
        &= -\frac{1}{x}
\end{align*}
A function in terms of $x$ only was found so this can be used to build an
integrating factor.

\begin{align*}
    \mu &= e^{\int -1/x \; dx} \\
        &= x^{-1}
\end{align*}

Multiply the integrating factor by the original equation and then
check to make sure it is now exact.
\begin{align*}
    \frac{(xy - 1)}{x} \; dx + \frac{ (x^2 - xy)}{x} \; dy &= 0 \\
    y - \frac{1}{x} \; dx + (x - y) \; dy &= 0 \\
    \frac{\partial{M}}{\partial{y}} = 1 &= \frac{\partial{N}}{\partial{x}}
\end{align*}
The modified equation is exact so now it can be solved normally.

Integrate each term seperately and then combine unique terms.
\begin{align*}
    \int \Big( y - \frac{1}{x} \Big) \; dx &= x y - \ln|x| \\
    \int \Big( x - y \Big) \; dy &= x y - \frac{y^2}{2} \\
    -\frac{y^2}{2} + x y - \ln|x| &= C
\end{align*}

\end{solution}

\question Solve the ODE:
\begin{align*}
    (xy + x) \; dx - (x^2 y^2 + x^2 + y^2 + 1) \; dy &= 0
\end{align*}

\vspace*{17em}

\begin{solution}
Separate.
\begin{align*}
    (xy + x) \; dx - (x^2 y^2 + x^2 + y^2 + 1) \; dy &= 0 \\
    x (y + 1) \; dx &= x^2 (y^2 + 1) + (y^2 + 1) \; dy \\
    \frac{(y+1)}{x} \; dx &= (y^2 + 1) (x^2 + 1) \; dy \\
    \frac{dx}{x(x^2 + 1)} &= \frac{y^2 + 1}{y + 1} \; dy
\end{align*}

In it's current form it is difficult to integrate.
The left hand side ($dx$) can be reduced using partial fractions.
And the right hand side can be reduced using polynomial long division.
The steps are not shown, just the result below.
\begin{align*}
    \Big( \frac{1}{x} - \frac{x}{x^2 + 1} \Big) \; dx &= \Big( y - 1 + \frac{2}{y+1} \Big) \; dy \\
    \ln|x| - \frac{1}{2} \cdot \ln|x^2 + 1| &= \frac{y^2}{2} - y + 2 \cdot \ln|y + 1| + C \\
    2 \ln|x| - \ln|x^2 + 1| &= y^2 - 2 y + 4 \cdot \ln|y + 1| + C
\end{align*}

\end{solution}

\question Assume that the rate at which radioactive nuclei decay is proporitonal to the
number of such nuclei that are present in a given sample.  In a certain sample
one-fourth of the original number of radioactive nuclei have undergone
disintegration in a period of 500 years. \\
\\
(a)  What fraction of the original radioactive nuclei will remain after 1000
years? \\
(b)  In how many years will one-half of the original number remain?

\vspace*{17em}

\begin{solution}
\begin{align*}
    \frac{dA}{dt} &= k \cdot A \\
    A &= A_0 \cdot e^{k \cdot t} \\
    \frac{3}{4} A_0 &= A_0 \cdot e^{k \cdot 500} \\
    k &= \frac{\ln(3/4)}{500} \\
    A &= A_0 \cdot e^{ \frac{\ln(3/4)}{500} \cdot t}
\end{align*}

a)
\begin{align*}
    t &= 1000 \;\; \mbox{years} \\
    z \cdot A_0 &= A_0 \cdot e^{ \frac{\ln(3/4)}{500} \cdot 1000} \\
    z &= 9/16
\end{align*}

b)
\begin{align*}
    z &= 1/2 \\
    t &= ? \\
    \frac{1}{2} &= e^{ \frac{\ln(3/4)}{500} \cdot t} \\
    t &= \frac{ \ln(1/2) \cdot 500 }{ \ln(3/4) } \\
      &= 1205 \;\; \mbox{years}
\end{align*}

\end{solution}

\end{questions}

\end{document}
