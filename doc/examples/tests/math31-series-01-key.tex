
\documentclass[fleqn]{article}
\usepackage{vmargin}
\usepackage{amsmath}
\parindent=0in

\begin{document}

\begin{flushleft}
math31-series-01 (1)\\
\today
\end{flushleft}


1, id: 6).  Does it converge?  And if so find the range of values.\\

\begin{align*}
\sum_{n=1}^{\infty} \left(3*x\right)^n\\
\end{align*}


\begin{verbatim}





\end{verbatim}

*** ANSWER ***


It is geometric with $r = 3*x$.

To converge the range of values of $x$ must satisfy $r < 1$.

\begin{align*}
\left| 3x \right| &< 1\\
\left| x \right| &< \frac{1}{3}\\
-\frac{1}{3} &< x < \frac{1}{3}\\
\end{align*}


2, id: 4).  Does it converge?  And if so find the exact value.\\

\begin{align*}
\sum_{n=1}^{\infty} \frac{4}{2^n}
\end{align*}


\begin{verbatim}





\end{verbatim}

*** ANSWER ***


It can be converted to geometric form.

\begin{align*}
&= 4 \sum_{n=1}^{\infty} \frac{1}{2^n}\\
&= 4 \sum_{n=1}^{\infty} \left(\frac{1}{2}\right)^n\\
\end{align*}

Since $r < 1$ it should converge.

\begin{align*}
&= 4 \frac{a}{1 - r}\\
&= 4 \frac{1}{1 - \frac{1}{2}}\\
&= 8\\
\end{align*}

Geometric, $r = 1/2$, converges to $8$.

3, id: 5).  Does it converge?  And if so find the range of values.\\

\begin{align*}
\sum_{n=1}^{\infty} \frac{x^n}{2^n}
\end{align*}


\begin{verbatim}





\end{verbatim}

*** ANSWER ***


It can be converted to geometric form.

\begin{align*}
&= \sum_{n=1}^{\infty} \left(\frac{x}{2}\right)^n\\
\end{align*}

Since it is geometric, it must satisfy $r < 1$ in order to converge.

\begin{align*}
\left| \frac{x}{2} \right| &< 1\\
\left| x \right| &< 2\\
-2 & < x < 2\\
\end{align*}


4, id: 8).  Does the series converge?\\

\begin{align*}
\frac{\ln2}{2} + \frac{\ln3}{3} + \frac{\ln4}{4} + \frac{\ln5}{5} + + +\\
\end{align*}

\begin{verbatim}





\end{verbatim}

*** ANSWER ***

\begin{align*}
\sum_{n=1}^{\infty} \frac{\ln(n+1)}{n+1}\\
\end{align*}

Let $f(x) = \frac{\ln(x+1)}{x+1}$.

Look at $\int_{1}^{\infty} \frac{\ln(x+1)}{x+1} dx$ .

\begin{align*}
& \int_{1}^{\infty} \frac{\ln(x+1)}{x+1} dx\\
&= \left[ \frac{(\ln(x+1))^2}{2} \right]_{1}^{\infty}\\
&= \infty - \frac{(\ln(2))^2}{2} = \infty
\end{align*}

Because the integral diverges so too will the series.

5, id: 1).  Does it converge?  And if so find the exact value.\\

\begin{align*}
\sum_{n=1}^{\infty}\frac{1}{n(n+3)}
\end{align*}


\begin{verbatim}





\end{verbatim}

*** ANSWER ***


The limit goes to zero so it should converge.
\begin{align*}
\lim_{n\rightarrow\infty}\frac{1}{n (n+3)}\\
\end{align*}

Using partial fractions of case 1 it converts to:
\begin{align*}
\frac{1}{3} \sum_{n=1}^{\infty}\frac{1}{n} - \frac{1}{n+3}
\end{align*}
which is in the form of a telescopic series

\begin{align*}
= & \left(1-\frac{1}{4}\right)+\left(\frac{1}{2}-\frac{1}{5}\right)+\left(\frac{1}{3}-\frac{1}{6}\right)+\left(\frac{1}{4}-\frac{1}{7}\right)...\left(\frac{1}{n-2}-\frac{1}{n+1}\right)+\left(\frac{1}{n-1}-\frac{1}{n+2}\right)+\left(\frac{1}{n}-\frac{1}{n+3}\right)\end{align*}

Seeing that the second term the first of the 3rd group away and continuing
this pattern results in:

\begin{align*}
\frac{1}{3}\lim_{n\rightarrow\infty}1+\frac{1}{2}+\frac{1}{3}-\frac{1}{n+1}-\frac{1}{n+2}-\frac{1}{n+3}
\end{align*}

which produces the exact value that the series converges to:

\begin{align*}
&= \frac{1}{3}*\frac{11}{6}\\
&= \frac{11}{18}
\end{align*}


6, id: 2).  Does it converge?  And if so find the exact value.\\

\begin{align*}
\sum_{n=1}^{\infty} \ln(\frac{n+1}{n})
\end{align*}


\begin{verbatim}





\end{verbatim}

*** ANSWER ***


The limit goes to zero so it should converge.

It is not geometric.

Let $f(x) = \ln((x+1)/x)$.
Look at $\int_{1}^{\infty} \ln((x+1)/x)$.

The integral of $f(x)$ goes to $\infty$ so the original series diverges.

7, id: 3).  Does it converge?  And if so find the exact value.\\

\begin{align*}
\sum_{n=1}^{\infty} \left(1+\frac{k}{n}\right)^{n}\\
when \;\; k 
eq 0
\end{align*}


\begin{verbatim}





\end{verbatim}

*** ANSWER ***


\begin{align*}
& \lim_{n \rightarrow \infty} \left(1+\frac{k}{n}\right)^{n}\\
&= e^k\\
&
eq 0\\
\end{align*}

Because the limit does not go to zero the series will not converge.


8, id: 7).  Does the series converge?\\

\begin{align*}
\frac{1}{2} + \frac{1}{5} + \frac{1}{10} + \frac{1}{17} + \frac{1}{26} + + +\\
\end{align*}

\begin{verbatim}





\end{verbatim}

*** ANSWER ***

\begin{align*}
\sum_{n=1}^{\infty} \frac{1}{n^2+1}\\
\end{align*}

Let $f(n) = \frac{1}{n^2+1}$.
Look at $\int_{1}^{\infty} \frac{1}{n^2+1} dn$.

\begin{align*}
& \int_{1}^{\infty} \frac{1}{n^2+1} dn\\
&= \left[ \arctan(n) \right]_{1}^{\infty}\\
&= \frac{\pi}{2} - \frac{\pi}{4} = \frac{\pi}{4}\\
\end{align*}

Because the integral converges so too will the series.


\end{document}
