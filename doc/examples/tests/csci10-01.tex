
\documentclass[fleqn]{article}
\usepackage{vmargin}
\parindent=0in

\begin{document}

\begin{flushleft}
csci10\\
\today
\end{flushleft}
TestID: 1
Wed, 28 Oct 2009 19:06:57 -0700
\\
\\

1).  Convert the number 5 to -5 using using binary 2s compliment.
Then convert -5 back to 5 in the same way.
\begin{verbatim}





\end{verbatim}

2).  Give the 2s compliment binary representation of -32767 (base 10)
\begin{verbatim}





\end{verbatim}

3).  Convert 12345 (base-10) to hexadecimal using the 'largest power of sixteen' algorithm.
\begin{verbatim}





\end{verbatim}

4).  Convert 33 (base-10) to hexadecimal using the 'largest power of sixteen' algorithm.
\begin{verbatim}





\end{verbatim}

5).  Give the 2s complement binary representation of 321 (base 10).
\begin{verbatim}





\end{verbatim}

6).  Give the 2s complement binary representation of -25 (base 10).
\begin{verbatim}





\end{verbatim}

7).  Give the 2s complement binary representation of -1234
\begin{verbatim}





\end{verbatim}

8).  Convert \verb+0xAA+ (base-16) to decimal.
\begin{verbatim}


\end{verbatim}

\end{document}
