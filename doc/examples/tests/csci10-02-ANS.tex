
\documentclass[fleqn]{article}
\usepackage{vmargin}
\parindent=0in

\begin{document}

\begin{flushleft}
csci10 (1)\\
\today
\end{flushleft}


1, id: 1).  Convert the number 5 to -5 using using binary 2s compliment.
Then convert -5 back to 5 in the same way.
\begin{verbatim}





\end{verbatim}

*** ANSWER ***

\begin{verbatim}
5 			(decimal)
0000_0101   (5 binary)
1111_1010	(invert all bits (NOT))
1111_1011	(add one, answer -5)

1111_1011   (-5 binary)
0000_0100   (invert all bits)
0000_0101   (add one, answer 5)
\end{verbatim}

2, id: 2).  Give the 2s compliment binary representation of -32767 (base 10)
\begin{verbatim}





\end{verbatim}

*** ANSWER ***

\begin{verbatim}
Convert 32767 to decimal.
32767				   1
16383		   	      11
8191		  	     111
4095		 	    1111
2047	   	      1 1111
1023      	     11 1111
511      	    111 1111
255     	   1111 1111
127   	     1 1111 1111
63	        11 1111 1111
31         111
15        1111
7       1
3      11
1  	  111
0 	 0111 1111 1111 1111

0111 1111 1111 1111		32767
1000 0000 0000 0000		invert
1000 0000 0000 0001		add 1, answer -32767
\end{verbatim}

3, id: 3).  Convert 12345 (base-10) to hexadecimal using the 'largest power of sixteen' algorithm.
\begin{verbatim}





\end{verbatim}

*** ANSWER ***

\begin{verbatim}
Table of powers of 16
---------------------
16^4	65536
16^3	4096
16^2	256
16^1	16
16^0	1

12345 (based 10)
3*16^3	12345-12288=57
0*16^2
3*16^1	57-48=9
9*16^0  

$3039
\end{verbatim}

4, id: 4).  Convert 33 (base-10) to hexadecimal using the 'largest power of sixteen' algorithm.
\begin{verbatim}





\end{verbatim}

*** ANSWER ***

\begin{verbatim}
Table of powers of 16
---------------------
16^2	256
16^1	16
16^0	1

33   (base 10)
2*16^1  33-32=1
1*16^0

$0021
\end{verbatim}

5, id: 5).  Give the 2s complement binary representation of 321 (base 10).
\begin{verbatim}





\end{verbatim}

*** ANSWER ***

\begin{verbatim}
2^9 = 512
2^8 = 256
2^7 = 128
2^6 = 64
2^5 = 32
2^4 = 16
2^3 = 8
2^2 = 4
2^1 = 2
2^0 = 1

1*2^8  321-256 = 65
0*2^7
1*2^6  65-64 = 1
0*2^5
0*2^4
0*2^3
0*2^2
0*2^1
1*2^0

0000 0001 0100 0001
\end{verbatim}

6, id: 6).  Give the 2s complement binary representation of -25 (base 10).
\begin{verbatim}





\end{verbatim}

*** ANSWER ***

\begin{verbatim}
decimal -> binary
1*2^4  25 - 16 = 9
1*2^3  9 - 8 = 1
0*2^2
0*2^1
1*2^0

0001 1001   (25)

First Method:
0001 1001   25
1110 0110        invert
1110 0111        add 1

1110 0111   -25  answer)

To check the answer, perform the procedure again
0001 1000        invert
0001 1001        add 1, OK

Second method:
0001 1001   25   express as a positive
---- ---1        start at right, write all bits through first 1
1110 011-        invert remaining bits
1110 0111        combine, answer, -25
1110 0111   -25
\end{verbatim}

7, id: 7).  Give the 2s complement binary representation of -1234
\begin{verbatim}





\end{verbatim}

*** ANSWER ***

\begin{verbatim}
1234 (decimal) -> binary

1*2^10  1234-1024 = 210
0*2^9
0*2^8
1*2^7   210 - 128 = 82
1*2^6   82 - 64 = 18
0*2^5
1*2^4   18 - 16 = 2
0*2^3
0*2^2
1*2^1   2 - 2 = 0
0*2^0   

0100 1101 0010  (1234 binary)

0000 0100 1101 0010
1111 1011 0010 1101  invert
1111 1011 0010 1110  add 1, answer -1234

check you answer
0000 0100 1101 0001 invert
0000 0100 1101 0010 add 1, 1234, OK
\end{verbatim}

8, id: 8).  Convert \verb+0xAA+ (base-16) to decimal.
\begin{verbatim}


\end{verbatim}

*** ANSWER ***

\begin{verbatim}
10*16^1 + 10*16^0 = 170 (decimal)
\end{verbatim}

\end{document}
